\subsection{mass2slice.pl\label{mass2slice_pl}\index{mass2slice.pl}}


Extract profile and cross section data from MASS2 plot
output files.

\subsection*{SYNOPSIS\label{mass2slice_pl_SYNOPSIS}\index{mass2slice pl!SYNOPSIS}}

perl \textbf{mass2slice.pl} [\textbf{-d}] [-a] [\textbf{-i}$|$\textbf{-j}] [\textbf{-t} \textit{indices}$|$\textbf{-l}] [\textbf{-o} \textit{output}]
\textit{file} \textit{variable} \textit{block} \textit{index} [\textit{block} \textit{index} ...]



perl \textbf{mass2slice.pl} \textbf{-p} [\textbf{-t} \textit{indices}$|$\textbf{-l}] [\textbf{-o} \textit{output}]
\textit{file} \textit{variable} \textit{block} \textit{i} \textit{j}

\subsection*{DESCRIPTION\label{mass2slice_pl_DESCRIPTION}\index{mass2slice pl!DESCRIPTION}}

\textbf{mass2slice.pl} is used to extract slices of data from the NetCDF
format MASS2 plot output file.  Slices can be made either
longitudinally (\textbf{-i}) or laterally (\textbf{-j}) across multiple blocks.
By default, all times are output separated by a blank line (useful for
plotting in gnuplot).

\subsection*{OPTIONS\label{mass2slice_pl_OPTIONS}\index{mass2slice pl!OPTIONS}}\begin{description}
\item[\textbf{-a}] \mbox{}

Average the variable across the profile or cross section to produce
one value.

\item[\textbf{-d}] \mbox{}

Place the date/time in the first two columns of the output.

\item[\textbf{-D}] \mbox{}

For \textbf{-i} and \textbf{-j} slices, do not output any points where the cell is
dry.  This option is ignored if there is no wet/drying information in \textit{file}.

\item[\textbf{-i}] \mbox{}

Extract a profile.  Distance is computed from upstream to downstream.
The \textit{index} is the lateral index.

\item[\textbf{-j}] \mbox{}

Extract a cross section. Distance is computed from the right to left
bank (looking downstream). The \textit{index} is the longitudinal index.

\item[\textbf{-p}] \mbox{}

Output a timeseries at a single point rather than a slice.  If this
option is used, both the \textit{i} and \textit{j} indices of the point (cell
actually) are expected to be on the command line.

\item[\textbf{-t} \textit{index}] \mbox{}

Select a specific time, specified by \textit{index}, to extract.  By default
all times are extracted, and their time stamp is placed in the output.

\item[\textbf{-l}] \mbox{}

Extract only the last time in the plot output file.  The \textbf{-t} is
ignored if this option is specified.

\item[\textbf{-o} \textit{output}] \mbox{}

Send data to \textit{output} instead of standard output.

\end{description}
\subsection*{EXAMPLES\label{mass2slice_pl_EXAMPLES}\index{mass2slice pl!EXAMPLES}}

This script is typically used to provide data for profile plotting.  The following gnuplot script will plot a water surface elevation profile from a three block domain:

\begin{verbatim}
 plot '<perl mass2slice.pl -t 1 -i plot.nc wsel 1 5 2 10 3 5' \
         using 3:4 with l 1, \
      '<perl mass2slice.pl -l -i plot.nc wsel 1 5 2 10 3 5' \
         using 3:4 with l 3, \
      '<perl mass2slice.pl -i plot.nc zbot 1 5 2 10 3 5' \
         using 3:4 with l 7
\end{verbatim}


Block 2 is either aligned differently or of a different resolution.
The resulting plot will show three curves: the initial conditions
(\texttt{-t 1, wsel}), the final conditions (\texttt{-l, wsel}), and the bottom
elevation (\texttt{zbot}).

\subsection*{SEE ALSO\label{mass2slice_pl_SEE_ALSO}\index{mass2slice pl!SEE ALSO}}

gnuplot(1)


%%% Local Variables: 
%%% mode: latex
%%% TeX-master: "main"
%%% End: 
